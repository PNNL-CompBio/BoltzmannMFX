\documentclass[12pt]{article}
\usepackage{xcolor}
\usepackage{graphicx}
%\usepackage{appendix}
%\usepackage{verbatim}
\usepackage[margin=1.0in]{geometry}
\DeclareGraphicsExtensions{.jpg,.jpeg,.png}
\renewcommand{\baselinestretch}{1.0}
\begin{document}
\newcommand{\dt}{\Delta t}
\newcommand{\Fvec}{\overline{F}}
\newcommand{\fvec}{\overline{f}}
\newcommand{\rvec}{\overline{r}}
\newcommand{\omvec}{\overline{\omega}}
\newcommand{\xvec}{\overline{x}}
\newcommand{\Xvec}{\overline{X}}
\newcommand{\yvec}{\overline{y}}
\newcommand{\Yvec}{\overline{Y}}
\newcommand{\Vvec}{\overline{V}}
\newcommand{\Rvec}{\overline{R}}
\newcommand{\Gvec}{\overline{G}}
\newcommand{\svec}{\overline{s}}
\newcommand{\Tvec}{\overline{T}}
\newcommand{\Lvec}{\overline{L}}
\newcommand{\vvec}{\overline{v}}
\newcommand{\nhat}{\hat{n}}
\newcommand{\rhvec}{\overline{\rho}}
\newcommand{\Dmat}{\overline{\overline{D}}}
\newcommand{\Rmat}{\overline{\overline{R}}}
\newcommand{\Omat}{\overline{\overline{1}}}
\newcommand{\Imat}{\overline{\overline{I}}}
\newcommand{\Hmat}{\overline{\overline{H}}}
\newcommand{\Lag}{\mathcal{L}}
\title{Simple Model of Fungal Growth on a Stationary Nutrient Medium}
\author{Written by a cast of thousands}
\maketitle

% Change table counter
\renewcommand{\thetable}{\Roman{table}}

\section{Model}
This document describes a fungal growth simulation implemented with the
BMX framework. Individual hyphae are modeled as connected cylinders that 
are designed to mimic the behavior of a fungal network that is growing by
elongation of tip elements and creation of new segments at the growth tips.
The hyphae cylinders are characterized by a length and a radius. Hyphae
that have reached a maximum size stop growing if they are in an interior
location while hyphae representing a growth tip can generate one or two new
hyphae at the free growth end of the cylinder. The tip splitting events that
result in two new hyphae generate branched structures similar to those
seen in real fungal systems. The simulation also supports fusion of growth
tips to interior portions of the existing fungal network.

The individual hyphae are represented by a single particle data structure inside
the AMReX framework. The hyphae are characterized by a position $\rvec$
representing the center of the cylinder, an orientation
specified by two angles $\phi$ and $\theta$ and the previously mentioned cylinder
length and radius. In addition, a large number of integer flags keep track of
bonding of the hyphae to other hyphae in the fungal network. These flags are
discussed in more detail below. Because the hyphae are cylinders that are rotationally
symmetric about the cylinder axis, only two angles are needed instead of the
customary three angles used to specify the orientation of an arbitrary shape
in three dimensions. The centers of the cylinder end-caps are located at positions
$\rvec_1$ and $\rvec_2$ and the angles $\phi$ and $\theta$ describe the unit vector
$\nhat = (\cos(\phi)\sin(\theta),\sin(\phi)\sin(\theta),\cos(\theta))$ pointing
from $\rvec_1$ to $\rvec_2$. For a cylinder of length $h$, the vectors $\rvec$,
$\rvec_1$ and $\rvec_2$ are related via
\begin{eqnarray*}
\rvec_1 & = & \rvec - \frac{h}{2}\nhat \\
\rvec_2 & = & \rvec + \frac{h}{2}\nhat
\end{eqnarray*}

\section{Chemistry Model}
The chemistry for this fungi simulation is similar to one constructed for looking
at bacteria or yeast and consists of a set
of equations that describe chemical dynamics of a simple system within
the fungal hyphae. Three chemical
constituents are considered, $A$, $B$
and $C$. The cell volume is also coupled to these constituents. $A$ and $C$
exist both inside and outside the cell (in the
agar medium). The concentrations outside the cell are denoted with the
subscript $out$ while those inside the cell are labeled
with the subscript $in$. The compound $B$ is only located inside the cell and
is impermeable to the cell membrane.
These species are related to each other via the following reactions:
\begin{eqnarray*}
A_{out} &\stackrel{k_{1},k_{-1}}{\longleftrightarrow}& A_{in} \\
A_{in} &\stackrel{k_2,k_{-2}}{\longleftrightarrow}& B_{in} + C_{in} \\
C_{in} &\stackrel{k_3,k_{-3}}{\longleftrightarrow}& C_{out}
\end{eqnarray*}
$A$ is only capable of undergoing a reaction inside the cell.  To distinguish
between concentrations of these components and absolute amounts of component, we use
brackets [ ]. In this notation, $[A_{out}]$ is the concentration of $A$ in a grid cell and
$A_{out}$ is the total amount (mass) of $A$ in a grid cell. There is also an
additional equation that converts $B_{in}$ into cell volume. This can be
expressed by the equation
\[
B_{in} \stackrel{k_g,k_v}{\longleftrightarrow} V_{cell}
\]
The rate constant $k_g$ describes how fast $B_{in}$ disappears from the system
and the rate constant $k_v$ describes how fast the cell volume $V_{cell}$
increases. In this model, the conversion of $B_{in}$ to volume is irreversible.

%The intracellular concentration of these species is governed by the set of equations
%\begin{eqnarray*}
%\frac{d [A_{out}]}{d t} &=&-k_1 [A_{out}]+k_{-1} [A_{in}] \\
%\frac{d [A_{in}]}{d t} &=&k_1 [A_{out}]-k_{-1} [A_{in}] - k_2 [A_{in}]
% + k_{-2}[B_{in}][C_{in}]\\
%\frac{d [B_{in}]}{d t} &=&k_2 [A_{in}] - k_{-2}[B_{in}][C_{in}] - k_g[B_{in}]\\
%\frac{d [C_{in}]}{d t} &=&k_2 [A_{in}] - k_{-2}[B_{in}][C_{in}]-k_{-3} [C_{in}]
%+ k_3 [C_{out}] \\
%\frac{d [C_{out}]}{d t} &=&k_{-3} [C_{in}]-k_3 [C_{out}]
%\end{eqnarray*}

The transport equations into and out of the hyphae from the external fluid are similar to those
examined in a previous paper.
After accounting for hyphae surface area and volume, as well as the fluid volume in the grid cell
containing the hyphae, the equations for transport in and out of the hyphae in a time increment
$\dt$ are
\begin{eqnarray*}
\frac{d [C_{in}]}{dt} &=& [C_{out}]\frac{A_{cell}}{V_{cell}}k_3 - [C_{in}]\frac{A_{cell}}{V_{cell}}
k_{-3} \\
\frac{d[C_{out}]}{dt} &=& [C_{in}]\frac{A_{cell}}{V_{grid}}k_{-3} - [C_{out}]\frac{A_{cell}}{V_{grid}}
k_{3}
\end{eqnarray*}
These equation are for component $C$. The variable $A_{cell}$ is the surface area of the hyphae,
$V_{cell}$ is the corresponding hyphe volume and $V_{grid}$ is the fluid volume in the grid cell.
Similar equations apply to component $A$.

Using these expressions for the change in concentration due to transport across the cell membrane, the
rate equations governing this system become
\begin{eqnarray*}
\frac{d [A_{out}]}{d t} &=&-k_1\frac{A_{cell}}{V_{grid}} [A_{out}]
+k_{-1}\frac{A_{cell}}{V_{grid}} [A_{in}] \\
\frac{d [A_{in}]}{d t} &=&k_1\frac{A_{cell}}{V_{cell}} [A_{out}]
-k_{-1}\frac{A_{cell}}{V_{cell}} [A_{in}] - k_2 [A_{in}]
 + k_{-2}[B_{in}][C_{in}]\\
\frac{d [B_{in}]}{d t} &=&k_2 [A_{in}] - k_{-2}[B_{in}][C_{in}]\\
\frac{d [C_{in}]}{d t} &=&k_2 [A_{in}] - k_{-2}[B_{in}][C_{in}]
-k_{-3} \frac{A_{cell}}{V_{cell}}[C_{in}]
+ k_3\frac{A_{cell}}{V_{cell}} [C_{out}] \\
\frac{d [C_{out}]}{d t} &=&k_{-3} \frac{A_{cell}}{V_{grid}}[C_{in}]
-k_3\frac{A_{cell}}{V_{grid}} [C_{out}]
\end{eqnarray*}
These reactions also act as sources and sinks for $A_{out}$ and $C_{out}$ in the diffusion
equations that govern transport in the agar growth medium.

An additional equation couples the concentration of $B_{in}$ to cell growth.
The growth is governed by the equation
\[
\frac{dV_{cell}}{dt} = V_{cell}k_v[B_{in}]
\]
The factor of $V_{cell}$ means that the total volume produced per unit time is
governed by the total amount of $B_{in}$ insided the cell. For the time being,
the rate constants are only non-zero if the segment is below the maximum size
limits. This means that growth only occurs at growth tips or, to a lesser
extent, segments that have just undergone a fusion event.

After updating the concentrations, an additional check is required to guarantee
that all components remain positive.
The BMX strategy for maintaining positive concentrations consists of two parts.
The first is to divide the available fluid in each grid cell evenly between each
of the particles located in the grid cell and the second is to check if any of the
components in the reaction schemes inside the particles goes negative.
Concentrations inside the particle or in the fluid can go negative because a
finite time step causes the system to overshoot a limit on the available reactant.
If this occurs, then the change in that component is modified so that the component
only goes to zero. This means that the change in any component that is coupled to
the changed component must also be adjusted. For example, for the reaction
\[
A \longrightarrow B + C
\]
we might calculate a change in $A$ over one time step, $\Delta A$, such that $A$
goes negative. This can be corrected by decreasing the change in $A$ to a value
$\Delta A'$ so that $A$ goes to zero instead. However, this means that the changes
in $B$ and $C$ must also be modified to reflect the smaller change in $A$.

Besides transport in and out of the cell coupled to diffusion in the external
growth medium, constituants can also be transferred between hyphae that are directly
bonded to each other. Hyphae that are linked together are separated by a permeable septum
that allows material to transfer between hyphae based on the concentration difference
across the septum. The flux of material between segments i and j is given by a relation
of the form
\[
f_{X,ij} = M_X ([X]_{in,i}-[X]_{in,i})
\]
where $f_{X,ij}$ is the flux of $X$ from segment i into segment j and $M_X$ is the mass
transfer coefficient for this process. This equation assumes perfect mixing inside the hyphae
so that the concentration at the septum is equal to the concentration within the cell.

Numerically, the rate equations are currently being handled using the following scheme
\begin{list}{$\bullet$}{}
\item There is an equilibration between the fluid and the cell over a time
increment $\dt/2$
\item The internal cell reaction takes place over a time increment $\dt$
using the updated internal concentrations from the previous step
\item Using the updated internal concentrations from the reaction step, the
concentrations are re-equilibrated with the outside fluid over a time interval
$\dt/2$
\end{list}
The first step is implemented by calculating the amount of material that flows
between the biological cell and the grid cell in which it is located.
The total amount of
material that flows into the cell in the interval $\dt$ is
\begin{eqnarray*}
\Delta A & = & \frac{\dt}{2} A_{cell}(k_1 [A_{out}]-k_{-1} [A_{in}]) \\
\Delta B & = & 0 \\
\Delta C & = & \frac{\dt}{2} A_{cell}(k_3 [C_{out}]-k_{-3} [C_{in}]) \\
\end{eqnarray*}
Based on this increment, the concentrations inside and outside the cell are
adjusted using the equations
\begin{eqnarray}
\nonumber
[A'_{in}] &=& [A_{in}] + \Delta A/V_{cell} \\
\nonumber
[B'_{in}]  &=& [B_{in}] \\
\nonumber
[C'_{in}] &=& [C_{in}] + \Delta C/V_{cell}
\end{eqnarray}
\begin{eqnarray}
\nonumber
[A'_{out}] &=& [A_{out}] - \Delta A/V_{grid} \\
\nonumber
[B'_{out}]  & = & [B_{out}] \\
\nonumber
[C'_{out}] &=& [C_{out}] - \Delta C/V_{grid}
\end{eqnarray}

The second step adjusts the concentration of reactants inside the cell based on
internal chemical reactions. Using a simple Euler scheme, the adjusted values of
the concentrations after the time increment $\dt$ are
\begin{eqnarray}
\nonumber
[A'_{in}] &=& [A_{in}] + \dt(-k_2 [A_{in}] + k_{-2} [B_{in}][C_{in}]) \\
\nonumber
[B'_{in}] &=& [B_{in}] + \dt(k_2 [A_{in}] - k_{-2} [B_{in}][C_{in}]) \\
\nonumber
[C'_{in}] &=& [C_{in}] + \dt(k_2 [A_{in}] - k_{-2} [B_{in}][C_{in}])
\end{eqnarray}
After adjusting the internal concentrations using the these equations, the
concentrations inside and outside the cell are again adjusted using the
equations in the first step.

In addition to exchanging chemical components with the external medium, chemical compounds can
diffuse through the septum between attached hyphae. Labeling the two hyphae with the indices $i$
and $j$, the flux $f_{Xij}$ from $i$ to $j$ through the hyphae for chemical species $X$ is
\[
f_{Xij} = M_{X}([X_{in}]_i-[X_{in}]_j)
\]
$M_{X}$ is a mass transfer coefficient for species $X$ and is assumed to be the same for all
hyphae. The actual amount of material transferred over a short time interval $\Delta t$ is the
flux multiplied by the area of the hyphae, $A_{ij}$, and the time interval $\dt$
\[
\Delta X_i = -\Delta X_j = -\Delta t A_{ij} M_{X}([X_{in}]_i-[X_{in}]_j)
\]
The area of the septum is taken as the average of the area of the end caps of the two joined
hyphae.

In this model, the exchanges between hyphae are completed after the chemistry and exchange
with the external medium are completed. The hyphae need to be updated in a
separate loop from the internal chemistry updates because pairs of hyphae are
needed for the exchange between different hyphae. The exchange of constituents
between hyphae again brings up an issue with maintaining positive
concentrationse. Theoretically, the transfer of material between hyphae should
be accomplished by solving a set of simultaneous equations for transfer between all hyphae in th
system. This would probably eliminate the issue of some concentrations going
negative but this is not practical. Apart from the fact that the AMReX framework does not support
setting up these kinds of equations in a distributed environment, the overhead of doing a
potentially large linear solve at each step is prohibitive. We opt for a simpler scheme that
should introduce minimal errors in the limit of small timesteps.

Each hyphae can be queried to find out how many other hyphae it is currently bonded to. This
number $M$ is between 1 and 4. The total amount of component $X$ is then divided up into $M$
separate portions $X/M$. This portion is then the maximum amount that can be exhanged with
another hyphae. If the exchange based on the mass transfer formula above exceeds this limit,
the amount exchanged is reduced to equal the portion size. For small enough time steps, the
exchanges will not hit the limit and this restriction has no effect. It is possible that an
exchange between one pair of hyphae that exceeds the limit could be counter-balanced by other
exchanges that don't result in a negative concentration. However, since exchanges are evaluated
completely independently of each other it is not possible to make use of this information.

\section{Cell forces}
The cell forces are calculated using a cubic interaction as described in Mathias
{\em et al}~\cite{Mathias}. The separation distance used in the force
calculation is the minimum separation distance between the two line segments
describing the two hyphae cylinders. The interactions are pairwise
additive, with the force between two particles $i$ and $j$ described by a function of the form
\[
\Fvec_{ij} = F_{ij}(r_{ij})\frac{\rvec_{ij}}{r_{ij}}
\]
where $F_{ij}(r_{ij})$ is a spherically symmetric function of the separation distance between the
centers of the two particles. For this model, the function $F(r)$ has the generic form
\begin{eqnarray*}
F(r) &=& -\mu(r-R_A)^2(r-R_S)\;\;\;\mbox{if}\;\;\;r<R_A \\
& = & 0\;\;\; \mbox{otherwise}
\end{eqnarray*}
The parameter $\mu$ is the stiffness and controls the magnitude of the repulsive and attractive
interactions. $R_S$ is the equilibrium separation value at which the net force on the particles is zero
and $R_A$ is the distance at which all pair interactions vanish. For $r<R_S$, the particles repel each
other while for $R_S<r<R_A$, the particles are attracted to each other. For this model, the values of
$R_S$ and $R_A$ are functions of the individual hyphae radii. If particle $i$ has radius $R_i$ and
particle $j$ has radius $R_j$ then the parameter $R_S$ is chosen as
\[
R_S = R_i+R_j
\]
The width of the attractive region, $R_A-R_S$, is chosen to have a fixed value $W_A$ for all particle
interactions. The value of $R_A$ is then
\[
R_A = R_i+R_j +W_A
\]
Note that the particle radii can increase over time, so even if two particles are in mechanical
equilibrium at some time $t$, they may start pushing against each other at a later time $t'$ as they
grow.

Particles also interact with the surface of the growth medium using a similar potential. In this
case the contribution to the force is only in the $z$-direction and is a function $H(z)$ of the height $z$ above
the growth medium surface. The force has the form
\begin{eqnarray*}
H(z) & = & -\xi (z-Z_A)^2(z-Z_S)\;\;\;\mbox{if}\;\;\;0<z<Z_A \\
& = & \xi Z_A^2Z_S\;\;\;\mbox{if}\;\;\;z<0 \\
& = & 0\;\;\;\mbox{otherwise}
\end{eqnarray*}
The parameter $\xi$ is the stiffness of the interaction, $Z_S$ is the equilibrium distance above the surface
and $Z_A$ is the point at which all interaction with the surface ends. Between $Z_S<z<Z_A$, particles are
attracted to the surface, while below $Z_S$ particles are repelled from the surface. Below the surface,
particles experience a constant upward force. This force is designed primarily to prevent particles from
getting accidentally pushed down into the growth medium. Similar to the pair interaction, for particle $i$,
the parameter $Z_S$ is given by
\[
Z_S = R_i
\]
and the parameter $Z_A$ is given by
\[
Z_S = R_i + W_Z
\]
where $W_Z$ is the width of the attractive region and is assumed to be independent of the particle's size.

The remaining calculation is to find the  minimum distance between two
line segments. The segments are defined by the line joining the centers of the two
end caps for each cylinder. If these two points are defined as $\rvec_1$ and $\rvec_2$
and the separation between them is $\rhvec = \rvec_2-\rvec_1$, then the line
segment $\Rvec(\tau)$ joining the endpoints is defined by
\[
\Rvec(\tau) = \rvec_1 +\tau\rhvec\;\;\;\;\;\;\;\mbox{for}\;\;\tau\in [0,1]
\]
For two cylinders $i$ and $j$ defined by the line segments $\Rvec_i(\tau_i)$ and
$\Rvec_j(\tau_j)$ we can find the minimum separation distance by minimizing
\begin{eqnarray*}
\chi & = & \left|\Rvec_i(\tau_i)-\Rvec_j(\tau_j)\right|^2 \\
     & = & \left|\rvec_{i1}-\rvec_{j1}\right|^2+2\tau_i(\rvec_{i1}-\rvec_{j1})\cdot\rhvec_i
          -2\tau_j(\rvec_{i1}-\rvec_{j1})\cdot\rhvec_j \\
     & + & \tau_i^2\left|\rhvec_i\right|^2 + \tau_j^2\left|\rhvec_j\right|^2
           - 2\tau_i\tau_j\rhvec_i\cdot\rhvec_j
\end{eqnarray*}
subject to the constraints $0\le\tau_i\le 1$ and $0\le\tau_j\le 1$.

Setting the gradients of $\chi$ with respect to $\tau_i$ and $\tau_j$ equal to
zero gives the coupled linear equations
\begin{eqnarray*}
-(\rvec_{i1}-\rvec_{j1})\cdot\rhvec_i & = & \tau_i\left|\rhvec_i\right|^2
                                        - \tau_j\rhvec_i\cdot\rhvec_j \\
(\rvec_{i1}-\rvec_{j1})\cdot\rhvec_j & = & - \tau_i\rhvec_i\cdot\rhvec_j
                                       + \tau_j\left|\rhvec_j\right|^2
\end{eqnarray*}
This can easily be solved in most cases for $\tau_i$ and $\tau_j$.
After a solution is found, it is necessary to check that the $\tau_i$ and $\tau_j$
satisfy the constraints. If one variable violates the constraints then the solution
can be found by setting the values of $\tau_i$ or $\tau_j$ to the limits 0 or 1 and
optimizing the remaining variable to find a minimum distance value. 
If both values violate the constraints then the minimum distance can be found
by searching the four pairs of values $\tau_i,\tau_j\in\{0,1\}$.

A corner case occurs when $\rhvec_i$ and $\rhvec_j$ are parallel. In this case the
determinant vanishes and there is no unique solution. An {\em ad hoc} solution can be
found by fixing $\tau_i$ to 0 or 1 and then minimizing $\tau_j$.

The force $\Fvec(R)$ acts on the two points $\Rvec_i(\tau_i)$ and
$\Rvec_j(\tau_j)$ on the cylinders but these sites do not necessarily correspond
to the sites at the ends of the cylinder.  The force must still be partitioned
between these. To do
this, we assume that $F(R)$ is generated from a potential $U(R)$. The force is
then
\[
\nabla_{\Rvec}U(R) = \Fvec(R)
\]
If we write $\Rvec_j(\tau_j)$ as
\[
\Rvec_i(\tau_i) = \rvec_i + \tau_i (\rvec_{i2}-\rvec_{i1})
\]
then the gradient of $U(R)$ with respect to $\rvec_{i1}$ is
\begin{eqnarray*}
\nabla_{\rvec_{i1}}U(R) & = & \frac{\partial U}{\partial R}\nabla_{\rvec_{i1}}R\
 = \frac{\partial U}{\partial R}\frac{\Rvec}{R} (1-\tau_i)
\end{eqnarray*}
Identifying $-\partial U/\partial R$ as $F(R)$ gives the final relation
\[
\fvec_{i1} = F(R)\frac{\Rvec}{R}(1-\tau_i)
\]
for the force acting on site 1 of segment $i$. The force acting on site 2 of segment
$i$ is just
\[
\fvec_{i2} = F(R)\frac{\Rvec}{R}\tau_i
\]
The expressions for the forces on the two sites of segment $j$ are similar.

\section{Hydrodynamic Drag}
To model hydrodynamic drag on the hyphae, each of the cites at the ends of the
hyphae is the center of a sphere. Both spheres are the same size
are chosen to represent
the hydrodynamic drag on the segment as forces from other segments push
it around. Each sphere is subject to a force, $\fvec_1$, $\fvec_2$, resulting in
a translational and rotational motion of the segment. For freely interacting
spheres, with no rigid constraints, the motion of the spheres can be described
by the matrix equation
\[
\left[\begin{array}{c}\vvec_1 \\ \vvec_2 \end{array}\right] =
\left[\begin{array}{cc} \Dmat_{11} & \Dmat_{12} \\
\Dmat_{21} & \Dmat_{22} \end{array}\right]
\cdot\left[\begin{array}{c} \fvec_1 \\ \fvec_2 \end{array}\right]
\]
The $3\times 3$ matrices $\Dmat_{ij}$ are diffusion matrices for hydrodynamically
interacting spheres. The details of these matrices can be found in Rotne and
Prager\cite{RP}. The vectors $\vvec_1$ and $\vvec_2$ are the
velocities of particles 1 and 2 that represent the filament segment.

%For our purposes, it is more convenient to express the equations of motion of the
%system in terms of net force and torque on the particle pair. To simplify this, we
%choose the two particles, separated by a distance $h$ to lie on the $x$-axis at
%positions $\rvec_1 = (h/2,0,0)$ and $\rvec_2 = (-h/2,0,0)$. The more general case
%can be derived by an appropriate rotation of the system. The torque $\Tvec$ and
%net force $\Fvec$ on the system can be written in terms of the forces $\fvec_1$
%and $\fvec_2$ and the positions $\rvec_1$ and $\rvec_2$ as
%\begin{eqnarray*}
%\Tvec & = & \rvec_1\times\fvec_1 + \rvec_2\times\fvec_2 \\
%\Fvec & = & \fvec_1 + \fvec_2
%\end{eqnarray*}
%Breaking these up into individual components gives the equations (for particles
%located at $\rvec_1$ and $\rvec_2$)
%\begin{eqnarray*}
%T_x & = & 0 \\
%T_y & = & \frac{h}{2}(f_{2z}-f_{1z}) \\
%T_z & = & \frac{h}{2}(f_{1y}-f_{2y}) \\
%F_x & = & f_{1x} + f_{2x} \\
%F_y & = & f_{1y} + f_{2y} \\
%F_z & = & f_{1z} + f_{2z}
%\end{eqnarray*}
%The equation for $F_x$ is actually a bit more complicated. The two particles
%cannot move relative to each other so an additional constraint would be
%\[
%f_{1x}-f_{2x} = 0
%\]
%indicating that there is no force pushing the two particles together. This could
%be considered as the result of a constraint force that acts to keep the two particles
%at a fixed distance. If the original $x$ components of the force acting on the two
%particles are considered as $f'_{1x}$ and $f'_{2x}$ and $F_x = f'_{1x} + f'_{2x}$
%then it is possible to use this value of $F_x$ and the two equations involving
%$f_{1x}$ and $f_{2x}$ to solve for $f_{1x}$ and $f_{2x}$.
%
%The equations for $\Tvec$ and $\Fvec$ can be inverted to solve for $\fvec_1$ and
%$\fvec_2$ in terms of $\Tvec$ and $\Fvec$. Writing this as a matrix equation gives
%\[
%\left[\begin{array}{c}f_{1x} \\ f_{1y} \\ f_{1z} \\ f_{2x} \\ f_{2y} \\ f_{2z}
% \end{array}\right] =
%\left[\begin{array}{cccccc}
%\frac{1}{2} & 0 & 0 & 0 & 0 & 0 \\
%0 & \frac{1}{2} & 0 & 0 & 0 & \frac{1}{h} \\
%0 & 0 & \frac{1}{2} & 0 & -\frac{1}{h} & 0 \\
%\frac{1}{2} & 0 & 0 & 0 & 0 & 0 \\
%0 & \frac{1}{2} & 0 & 0 & 0 & -\frac{1}{h} \\
%0 & 0 & \frac{1}{2} & 0  &\frac{1}{h} & 0
%\end{array}\right]\cdot
%\left[\begin{array}{c} F_x \\ F_y \\ F_z \\ T_x \\ T_y \\ T_z
%\end{array}\right]
%\]
%
%It is also useful to decompose the velocities $\vvec_1$ and $vvec_2$ into a velocity
%of the center of mass, $\Vvec$, and an angular velocity, $\omvec$. The center of mass
%velocity is given by the relation
%\[
%\Vvec = \frac{1}{2}(\vvec_1+\vvec_2)
%\]
%The angular velocity must satisfy the relations
%\[
%(\vvec_1-\Vvec) = \omvec\times\rvec_1
%\]
%Since $\rvec_2=-\rvec_1$ it follows that $\vvec_1-\Vvec=-(\vvec_2-\Vvec)$. This is
%is equivalent to saying the relative motion of particles 1 and 2 is zero, which is
%required if the particles remain separated by a fixed distance. Using the
%coordinates for $\rvec_1$ and $\rvec_2$ as above, the components of $\omvec$ can be
%written as
%\begin{eqnarray*}
%v_{1x}-V_x & = & 0 \\
%v_{1y}-V_y & = & \frac{h}{2} \omega_z \\
%v_{1z}-V_z & = & -\frac{h}{2} \omega_y
%\end{eqnarray*}
%The equations generated by $\vvec_2$ are essentially that same as the equations
%generated by $\vvec_1$. The $x$ component of $\omvec$ is indeterminate, but can be
%set to 0 (the forces in this model are applied to points, so there is no torque
%about the $x$ axis). These equations can be solved for $\Vvec$ and $\omvec$ to get
%the matrix equations
%\[
%\left[\begin{array}{c}V_x \\ V_y \\ V_z \\ \omega_x \\ \omega_y \\ \omega_z
% \end{array}\right] =
%\left[\begin{array}{cccccc}
%\frac{1}{2} & 0 & 0 & \frac{1}{2} & 0 & 0 \\
%0 & \frac{1}{2} & 0 & 0 & \frac{1}{2} & 0 \\
%0 & 0 & \frac{1}{2} & 0 & 0 & \frac{1}{2} \\
%0 & 0 & 0 & 0 & 0 & 0 \\
%0 & 0 & -\frac{1}{h} & 0 & 0 & \frac{1}{h} \\
%0 & \frac{1}{h} & 0 & 0  &-\frac{1}{h} & 0
%\end{array}\right]\cdot
%\left[\begin{array}{c} v_{1x} \\ v_{1y} \\ v_{1z} \\ v_{2x} \\ v_{2y} \\ v_{2z}
%\end{array}\right]
%\]
%

From Rotne and Prager, the diffusion tensor for a pair of
particles of radius $a$ is
\begin{eqnarray*}
\Dmat_{ii} & = & \frac{kT}{6\pi\mu a}\Omat \\
\Dmat_{ij} & = & \frac{kT}{8\pi\mu r_{ij}^3}\left[\left(\Omat r_{ij}^2 +\rvec_{ij}
\rvec_{ij}\right) +\frac{2a^2}{r_{ij}^2}\left(\frac{1}{3}\Omat r_{ij}^2
-\rvec_{ij}\rvec_{ij}\right)\right] \;\;\;\; \mbox{if}\;\; r_{ij} > 2a \\
\Dmat_{ij} & = & \frac{kT}{6\pi\mu a}\left[\left(1-\frac{9}{32}\frac{r_{ij}}{a}\right)
+\frac{3}{32}\frac{\rvec_{ij}\rvec_{ij}}{ar_{ij}}\right] \;\;\;\; \mbox{if} \;\; r_{ij}<2a
\end{eqnarray*}
Assuming that the two particles lie at points $(h/2,0,0)$ and $(-h/2,0,0)$,
the equations for the off-diagonal matrices reduce to
\begin{eqnarray*}
(\Dmat_{ij})_{xx} & = & \frac{kT}{4\pi\mu h^3}\left[h^2-\frac{2}{3}a^2\right]
 \;\;\;\;\mbox{if}\;\;h>2a \\
(\Dmat_{ij})_{yy,zz} & = & \frac{kT}{8\pi\mu h^3}\left[h^2+\frac{2}{3}a^2\right]
 \;\;\;\;\mbox{if}\;\;h>2a \\
(\Dmat_{ij})_{xx} & = & \frac{kT}{6\pi\mu a}\left[1-\frac{3}{16}\frac{h}{a}\right]
 \;\;\;\;\mbox{if}\;\;h<2a \\
(\Dmat_{ij})_{yy,zz} & = & \frac{kT}{6\pi\mu a}\left[1-\frac{9}{32}\frac{h}{a}\right]
 \;\;\;\;\mbox{if}\;\;h<2a
\end{eqnarray*}
All other matrix elements are zero in this orientation. Note that these matrix
elements are continuous at the contact value of $h=2a$. This formula can be used to
calculate the diffusion tensor in the frame of reference where the cylinder is located
along the $x$ axis and then the tensor can be rotated back to the actual orientation.

Two remaining details are to choose is to choose a value for the radius $a$ and
to account for the fixed separation of the ends of the segment as it translates
and rotates. For these simulations
we choose the radius $a$ so that the total area of the two spheres is equal to the
area of the area of the cylinder represented by the segment.
To account for the rigid body motion of the two spheres at the ends of the
cylinder, we subtract out half of the relative velocity of the spheres towards each
other from each sphere. If the spheres at $\rvec_1$ and $\rvec_2$ have relative
 velocity
$\nhat_{12}\cdot(\vvec_1-\vvec_2)$ then the modified velocities are
\begin{eqnarray*}
\vvec'_1 & = & \vvec_1 - \frac{1}{2}\nhat_{12}(\nhat_{12}\cdot(\vvec_1-\vvec_2)) \\
\vvec'_2 & = & \vvec_2 + \frac{1}{2}\nhat_{12}(\nhat_{12}\cdot(\vvec_1-\vvec_2))
\end{eqnarray*}

The simulation code actually uses the translational velocity of the center of mass and
the angular rotational velocity, $\omvec$, of the cylinder to update the position of the system.
The two fictitious particles used to  calculate the hydrodynamic drag are assumed to
have the same mass, which leads immediately to the relation
\[
\Vvec = \frac{\vvec_1+\vvec_2}{2}
\]
between the center of mass velocity $\Vvec$ and the individual particle velocities.
Note that the translational velocity is not affected by subtracting out the
relative velocity of the two spheres.

The angular velocity is more complicated. The relationship between angular velocity
and the velocity of the individual particles in the cylinder is given by
\[
\Imat\cdot\omvec = \Lvec
\]
where $\Lvec$ is the angular momentum
\begin{eqnarray*}
\Lvec & = & \sum_i m\rvec_i\times\vvec_i\\
& = & \frac{m}{2}\rvec_{12}\times(\vvec_1-\vvec_2)
\end{eqnarray*}
and $\Imat$ is the moment of inertia tensor
\begin{eqnarray*}
\Imat & = & \sum_i m (\Omat r_i^2-\rvec_i\rvec_i) \\
& = & \frac{m}{4}(\Omat\; 2r_{12}^2-2\rvec_{12}\rvec_{12}) \\
& = & \frac{m}{2}(\Omat r_{12}^2-\rvec_{12}\rvec_{12}) \\
\end{eqnarray*}
Note that in all these equations, the particles are assumed to be located at
$\pm \rvec_{12}/2$.

If the cylinder is oriented parallel to the $x$ axis, then the moment of inertia tensor
has the simple form
\[
\Imat = \frac{mh^2}{2}\left[\begin{array}{ccc}
0 & 0 & 0\\
0 & 1 & 0\\
0 & 0 & 1\\
\end{array}\right]
\]
Note that this has a zero eigenvalue so that it is not invertible. However, in this
orientation, the $x$ component of $\Lvec$ also vanishes, implying that $\omega_x$ can be
set to any value. For convenience, we set it equal to zero.
%The angular momentum in this
%orientation is
%\[
%\Lvec = \frac{mh}{2}(0,\;-v'_z,\;v'_y)
%\]
%where $\vvec'$ is the vector $\vvec_1-\vvec_2$ after rotating to the frame where
%$\rvec_{12}$ is parallel to the $x$ axis. In this frame, the remaining two components
%of $\omvec$ are given by 
%\begin{eqnarray*}
%\omega_y & = & -v'_z/h \\
%\omega_z & = & v'_y/h
%\end{eqnarray*}
The remaining components of $\omvec$ can be computed from the rotated angular momentum
$\Lvec'$ using
\begin{eqnarray*}
\omega_y & = & 2L_y'/h^2 \\
\omega_z & = & 2L_z'/h^2 \\
\end{eqnarray*}

These equations have all relied on the fact that $\rvec_{12}$ is oriented along the
$x$ axis. To use these equations in practice, it is necessary to construct the rotation
that orients $\rvec_{12}$ along the $x$ axis. Assume polar coordinates with polar
angle $\phi$ and azimuthal angle $\theta$ such that $\theta = 0$ implies that a vector
is parallel to the $z$ axis and $\phi = 0$, $\theta = \pi/2$ implies that a vector is
parallel to the $x$ axis. The vector $\rvec_{12}$ is described by the coordinates
$h(\cos\phi\sin\theta,\sin\phi\sin\theta,\cos\theta)$. The matrix to rotate the $x$ axis
to this orientation is
\[
\Rmat=\left[\begin{array}{ccc}
\cos\phi\sin\theta & -\sin\phi & -\cos\phi\cos\theta \\
\sin\phi\sin\theta & \cos\phi & -\sin\phi\cos\theta \\
\cos\theta & 0 & \sin\theta
\end{array}\right]
=\left[\begin{array}{ccc}
\cos\phi & -\sin\phi & 0 \\
\sin\phi & \cos\phi & 0 \\
0 & 0 & 1\end{array}\right]
\left[\begin{array}{ccc}
\sin\theta & 0 & -\cos\theta \\
0 & 1 & 0 \\
\cos\theta & 0 & \sin\theta
\end{array}\right]
\]
The inverse rotation is just the transpose of this matrix.

A final rotation is required to rotate the about the $x$ axes to align the rotational
velocity with the $z$ axis. The rotation angle can be found by normalizing $\omvec$
and then finding the angle it makes with the $z$ axis (this should just be the
$\cos^{-1}$ of the $z$ component of the normalized rotational velocity). The
orientation of the segment can then be updated in this configuration and the segment
is then returned to it original orientation by applying the inverse rotations.

\section{Maintaining Connectivities Between Segments}
To model filamentous fungi, it is necessary to maintain bonds between
different linear segments. The end of one segment is bonded to the end of another
segment so that some sort of fixed relation can be maintained (A is bonded to B),
even if the two segments are hosted on different processors. The existing code
base does not support a bond list that can be used to maintain bonds
between different segments. In order to create a mechanism that allows us to
established fixed relations between different segments, the following is assumed
to be true
\begin{list}{$\bullet$}{}
\item Any segment B that is bonded to segment A is located within a finite
distance of A and lies in the neighbor list of A (the neighbor list is based
exclusively on position and is supported by the code). The same is true of A with
respect to B.
\item There is a finite maximum number of segments that can be bonded to a
central segment (in our case the maximum is four)
\item At most one bond can exist between two segments
\item A unique identifier can be created for each new segment that is added to the
system and is maintained over the life of the simulation. This identifier is also
restored if the calculation is restarted from a checkpoint file.
\end{list}
The last item on this list is a bit more complicated than it might initially appear.
The unique identifier in our implementation is, in fact, an integer pair. The first
integer is a unique index associated with the process that created the segment and
the second index is the rank of the process. This combination is unique across all
processes and can be generated without communication between processes.

The four events that can generate new particles are
\begin{list}{$\bullet$}{}
\item Segment generation from a growth tip.
\item Tip splitting of a growth tip to generate two new segments.
\item Side branching from an interiour segment.
\item Fusion of a growth tip to an interior segment. The interior segment splits
into two segments at the point at which the growth tip fuses.
\end{list}
Each of these events can generate one or two segments and it is necessary for internal data
on all particles to be updated to reflect the formation of new bonds and, in some cases,
the removal of old bonds. All of this is complicated by the fact that modifications to
particle data need to be done in a particular way in order for modifications to be made
permanent. This can be seen when looking at loops over neighbor pairs.

\begin{figure}
\centering
\includegraphics[width=4.0in,keepaspectratio=true]{side_branch}
\caption{\label{fig:side_branch} Schematic diagram of side branching}
\end{figure}

A typical loop over neighbors has the structure
\[
\sum_{i}\sum_{j\in Nghb(i)}
\]
The sum over $j$ may also have a restriction that the interaction is only evaluated
for $i<j$ to avoid duplicate interactions. The issue that arises, is that any
modifications that are identified in a loop over particles of this sort can only
be permanently applied to the particles indexed by $i$. The particles in the neighbor list
indexed by $j$ are copies of the original particle data, which may reside on other
processors, and changes made to the copies will disappear when the neighbor list is rebuilt.

To maintain and update the connectivity information, the following information is maintained
on each segment
\begin{list}{$\bullet$}{}
\item the number of segments that this segment is bonded to. The maximum number is 4.
\item a list of ids for the segments that are bonded to this segment (up to 4)
\item a list of cpus for the segments that are bonded to this segment (up to 4)
\item a list describing the site (1 or 2) at which the bonded segments are connected (up to 4)
\item an integer enumeration specifying whether a segment is a growth tip or not
\item a flag indicating that this segment is fusing to another segment
\item a flag indicating that this segment is going to split in a fusion event
\item a flag indicating that this segment was created during a fusion event
\item an integer parameter that is used to determine which end of the segment a bond fix will
attach to
\item the id of the segment that this particle is fusing to
\item the cpu of the segment that this particle is fusing to
\item a list of segment ids that were deleted from this segment (up to 2)
\item a list of segment cpus that were deleted from this segment (up to 2)
\end{list}
Each segment has a site at either end that is capable of forming bonds with other segments.
These sites are labeled 1 and 2 and each site can be bonded to a maximum of two other segments
for a maximum of four bonds per segment. The last eight items in the list are only used to
support fusion of a growth tip to another segment. Not all of the parameters are necessarily
used simultaneously by all segments involved in the fusion event.

A major function of the simulation is maintaining these connectivities and it is complicated
to a significant degree by the behavior of the loops over neighbors. Generally, when new
segments are created, they are spawned from a parent segment. At the moment of creation,
the permanent copy of both the parent and the newly created child segment are available to
to the program and modifications to their data will be preserved. Thus, when a growth tip
generates a new tip or splits into two new tips, all the permanent information for both the
parent segment and the new segments are available and all bonding information can be updated
at the same time.

The situation is more complicated for side-branching from an interior segment. This
is illustrated in Figure \ref{fig:side_branch}. In the figure, the interior segment A is bonded
to segment B and a side branch forms in segment A at the end that is bonded to segment B. The
resulting segment C is bonded to both A and B. At the moment when segment C forms, both the
permanent data for A and C are available, but not the data for B. A can set its internal data
so that it is bonded to C and C, because it has access to A's internal data and can see that the
end bonded to C is also bonded to B, can set its
internal data so that it it is bonded to A and B. However, at this point B's bonding data does
not list C as a bonding partner.
This must be fixed up later when the loop to calculate forces is performed. Furthermore, to
guarantee that segment C's permanent data is modified, the restriction $i<j$ cannot be used.
Before evaluating the forces between a pair of segments, each segment in the pair
checks to see if it bonded to the other segment. If it is not, then it checks the bonding
information in the neighbor to see if the neighbor is bonded to it. If the neighbor is bonded
then the segment updates its bonding information to list the neighbor as a bond. In the case
of side branching, segment C is bonded to segment B but B is not bonded to segment C.
In the force loop, segment B will check the bonding information of segment C and find that
segment C lists B as a bonding partner. Based on this, segment B will update its bond
information to include C.

\begin{figure}
\centering
\includegraphics[width=4.0in,keepaspectratio=true]{fusion}
\caption{\label{fig:fusion} Schematic diagram of a growth tip fusing with an interior segment}
\end{figure}

Fusion of a growth tip with an interior segment requires the addition of several more loops
over neighbors in order to maintain an accurate record of connectivity. A schematic diagram
of a tip fusion event is shown in Figure \ref{fig:fusion}. When a tip fuses to another segment,
the other segment splits at the point where the fusion occurs and two new segments are formed.

Computationally, a tip fusion event occurs in four separate steps:
\begin{enumerate}
\item In a double loop over all segments and their neighbors, a tip identifies another segment
that it is going to fuse with. The tip labels itself as fusing and stores the id and cpu of the
fusing partner. In Figure \ref{fig:fusion} segment A identifies segment B as a fusion partner
and marks itself accordingly. Segment A also adds segment B to its list of bonding partners,
\item A second double loop over all segments and their neighbors is performed and each segment
checks its neighbors to see if it is a fusing tip and if it has marked the segment as a fusing
partner. If it finds that it is a fusing partner, the segment marks itself as splitting and
stores the id and cpu of the tip that it is fusing with. It also adds the tip as a bonding
partner. In the example shown in Figure \ref{fig:fusion}, segment B finds that segment A has
marked itself as fusing and lists segment B as its fusing partner. Segment B marks itself as
splitting, stores the id and cpu of segment A and adds A as a bonding partner.
\item A third loop over segments is performed and all segments marked for splitting are split
into two segments. In the example in Figure \ref{fig:fusion}, segment B generates segment D.
Segment D is marked as being a new segment derived from a fusion event.
The split is done in a consisistent way so that the new segment is generated at the end of the
original segment corresponding to bonding site 2. Because of this, the bonds at site 2 on the
original segment B that bond to segment C are no longer valid and need to be replaced with bonds to D.
Similarly, bonds to A, B and C can be added to D at this point since D has access to B's
connectivity data. However, A does not list D as a bonding partner and C needs to modify its
bonding data to remove the bonds to B and add bonds to D. To do this, B stores the id and cpu
of the bond to C that was deleted. If B is attached to two segments at site 2, then the ids and
cpus of both segments are stored.

An additional complication is that when A and the
segments that were bonded to B at site 2 update their bond lists in the force loop, they
won't know to which site to add the new bonds. This can be handled by the integer bond fix
parameter. When segment A determines that it is fusing to segment B, it sets its copy of the
bond fix parameter to the value of 1 or 2 corresponding to the end of the segment that is
fusing to B. The segments that were formerly bonded to B, in this case C, will set this
parameter in the next step.

\item A fourth double loop including neighbors is performed to fix up the bond on segment C.
Bonds between segments A and D and segments C and D will get fixed in the force loop when
bonds are recognized between segments A and C and segment D. Segment B will still show up as
a neighbor of
segment C in the cleanup loop. However, segment B will be marked as having just been split and
will list a
bond to C as having been deleted. This is a signal to C to remove B as a bonding partner. C
will also have information on which site B is attached and can use this to set the value of
the bond fix flag to the site at which the bond to B was attached.
\end{enumerate}
Once these operations have been completed, the force loop will fix up any remaining
bonds that are only defined on one segment of a bonding pair and all other fusion parameters are
reset to a state that can initiate new events. For a side branching event, the bond fix parameter
will be set to zero. If a zero value for the bond fix parameter is encountered in the force loop,
then the segment that is
updating its bond list checks which site the first bond in its bond list is using and then sets
the bond fix parameter to the other site. This is enough to guarantee that the new bond is
attached to the correct site.
\bibliographystyle{unsrt}
\bibliography{bmx}
\end{document}
